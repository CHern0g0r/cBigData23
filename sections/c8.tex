\section{Гетерогенные УБД. Greenplum. Архитектура. Особенности, сценарии использования. Достоинства и недостатки.}

\subsection*{Гетерогенные УБД}

СМ 5

\subsection*{Greenplum}

\D{
    Greenplum – массивно-параллельная реляционная СУБД для
    хранилищ данных с гибкой горизонтальной масштабируемостью
    и столбцовым хранением данных на основе PostgreSQL.
}

Greenplum представляет собой несколько взаимосвязанных
экземпляров базы данных PostgreSQL, объединенных в кластер
по принципу массивно-параллельной архитектуры (Massive
Parallel Processing, MPP) без разделения ресурсов (Shared
Nothing).

При этом каждый узел кластера, взаимодействующий с
другими для выполнения вычислительных операций, имеет
собственную память, операционную систему и жесткие диски.

Мастер-сервер (Master host), где развернут главный инстанс
PostgreSQL (Master instance). Это точка входа в Greenplum, куда
подключаются клиенты, отправляя SQL-запросы.

Мастер координирует свою работу с сегментами – другими
экземплярами базы данных PostgreSQL.

Мастер распределяет нагрузку между сегментами, но сам не
содержит никаких пользовательских данных – они хранятся
только на сегментах.

Есть копия мастера.

Сервер-сегмент (Segment host), где хранятся и
обрабатываются данные. На одном хост-сегменте содержится 2-8
сегментов Greenplum – независимых экземпляров PostgreSQL с
частью данных. Сегменты Гринплам бывают основные (primary) и
зеркальные (mirror).

Primary-сегмент обрабатывает локальные данные и отдает
результаты мастеру. Каждому primary-сегменту соответствует свое
зеркало (Mirror segment instance), которое автоматически
включается в работу при отказе primary.

Интерконнект (interconnect) – быстрое обособленное сетевое
соединение для связи между отдельными экземплярами
PostgreSQL.

Применение:
\begin{itemize}
    \item системы предиктивной аналитики и регулярной отчетности по
    большим объемам данных;
    \item построение озер и корпоративных хранилищ данных;
    \item разработка аналитических моделей по множеству
    разнообразных данных, например, для прогнозирования оттока
    клиентов (Churn Rate)
\end{itemize}

