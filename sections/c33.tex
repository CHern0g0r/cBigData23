\section{Kubernetes. ConfigMap. Resources. QoS. Монтирование Docker volume.}

\D{
    ConfigMap - объект предназначенный для хранения пар ключ/значение.

    Позволяет упростить расширение и изменение пода.
}

\D{
    Ресурсы - все, что рассматривали до этого (поды, переменные среды...)
}

Используеются для:
\begin{itemize}
    \item Предотвращение негативного влияния на узел из-за неожиданного поведения
    \item Эффективное использование ресурсов
    \item Улучшение планирования
\end{itemize}

Можно задавать всякие лимиты.

\D{
    Quality of Service (QoS) - класс, определяющией планирование
    пода в кластере и приоритет выселения. QoS класс используется
    для назначения узла поду.
}

QoS классы:
\begin{itemize}
    \item Guaranteed
    
    Запрашиваемые и макс ресурсы равны или указаны только макс.
    \item Burstable
    
    Запрашиваемые и макс ресурсы разные или указаны только запрашиваемые.
    \item BestEffort
    
    Запрашиваемые и макс ресурсы не заданы.
\end{itemize}

\subsection*{Pods'n'Volumes}

Можно монтировать докер образ в под.

HostPath прописывается в конфиге.
