\section{Kubernetes. HostPath. PersistentVolume, PersistentVolumeClaim.}

HostPath прописывается в конфиге пода.

Типы HostPath:
\begin{itemize}
    \item Dir
    \item File
    \item Socket
    \item FileOrCreate
    \item DirOrCreate
    \item CharDevice
    \item BlockDevice
\end{itemize}

Проблемы:
\begin{itemize}
    \item Небезопасно
    \item Данные хранятся только на одном узле
    \item Непредсказуемое поведение
    \item Проблемы с разрешениями
\end{itemize}

\D{
    PersistentVolume - неизменяемое представление, доступное для взаимодействия
    как с реальным хранилищем, оборачивающее внутреннее представление хранилища
    (NFS, DFS, LocalFS...).

    PersistentVolumeClaim - требования, которые предоставляются контейнером
    к PersistentVolume, необходимым для подключения.
}

Типы PersistentVolume:
\begin{itemize}
    \item awsElasticBlockStore
    \item asureDisk
    \item cephfs
    \item csi
    \item gcePersistentDisk
    \item glusterfs
    \item local
    \item nfs
    \item ...
\end{itemize}

PV может быть одновременно замонтирован с использованием только одного
режима доступа:
\begin{itemize}
    \item ReadWriteMany
    \item ReadWriteOnce
    \item ReadWriteOncePod
    \item ReadOnlyMany
\end{itemize}
