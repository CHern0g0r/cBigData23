\section{Модели хранения данных. Теорема CAP. ACID свойства.}

Основные категории систем хранения данных (схд):
\begin{itemize}
    \item Файловая
    \item Объектная
    \begin{itemize}
        \item SQLike. Могут использоваться для BigData. Разница в представлении.
        \item NoSQL (BigData). Придуманы специально для проблем реляционок.
    \end{itemize}
\end{itemize}

\T[CAP]{
    В любой реализации распределенной обработки данных может быть
    достигнуто не более 2 из свойств:
    \begin{itemize}
        \item Согласованность - данные на всех узлах в каждый момент не противоречивы.
        \item Доступность - любой запрос обработан корректно.
        \item Устойчивость к разделениям - система функционирует при частичной потере
        компонентов
    \end{itemize}
}

\D{
    ACID - требования к транзакционной системе.
    \begin{itemize}
        \item Атомарность - ни одна транзакция не будет частично зафиксирована.
        \item Согласованность - любая операция приводит только к правильному результату.
        \item Изоляция - параллельные операции не влияют на результат текущей.
        \item Долговечность - в случае сбоя системы результат успешной
        операции должен сохраняться после восстановления.
    \end{itemize}
}

Типы систем:
\begin{itemize}
    \item CP - сохраняем возможность читать, но не менять.
    \item AP - разрешаем операции, но не гарантируем согласованность.
    \item CA
\end{itemize}
