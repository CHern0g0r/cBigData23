\section{Типы обработки данных. Grid, WMS, All-Pairs абстракции,}

Типы обработки данных:
\begin{itemize}
    \item Пакетная.
    Набор достаточного количества данных собирается со временем, затем проводистя
    анализ. Можно выбрать анализ нговой порции информации.
    \begin{itemize}
        \item[+] низкие капитальные затраты
        \item[+] большой объем
        \item[-] высокая задержка
        \item[-] невозможность обнаружения проблемы на месте.
    \end{itemize}
    \item Микропакетная.
    \item Потоковая.
    Данные можно анализировать сразу при получении.
    \begin{itemize}
        \item[+] меньше задержка
        \item[-] дорого
        \item[-] сложно
        \item[-] важность отказоустойчивости
    \end{itemize}
\end{itemize}

\subsection*{Абстракции (пакетной???)}

\subsubsection*{Grid}

Gfarm - реализация системы обработки большого массива данных.

Gfarm составляющие:
\begin{itemize}
    \item параллельная файловая система
    \item узлы
    \item исполнение
\end{itemize}

\subsubsection*{WMS}

*Система управления рабочими процессами*

Традиционной WMS является DAG, хотя сейчас присутствуют циклы и условия.

Благодаря развитию механизмов планирования решается вопрос оптимизации приложений
требующих больших объемов данных.

\subsubsection*{All-Pairs}

Имеет скорее академический интерес.

Идея: система принимает функцию $F: A \times B \to C$ и возвращает матрицу
$C^{A \times B}$.

Важна реализация.